\section*{Introduzione}

 

Appunti ordinati, con approfondimenti passo-passo, del corso di Telecomunicazioni per il corso di laurea in Ingegneria Elettronica 
presso l’Università Politecnica delle Marche per l'A.A. 2024/2025. \newline
        
Per quelli che seguono il nuovo corso di Ingegneria elettronica: \\
\url{https://www.univpm.it/Entra/Offerta_formativa_1/Corso_di_laurea_triennale_in_Ingegneria_Elettronica_e_delle_Tecnologie_Digitali} \\
cioè la Laurea in Ingegneria Elettronica e delle Tecnologie Digitali, 
il corso prende il nome di Telecomunicazioni digitali. \newline 


Le fonti degli appunti sono le seguenti: 

\begin{itemize}
    
    \item Slide del corso del prof Franco Chiaraluce 
    "Telecomunicazioni digitali A.A. 2024/2025" 
    
    \item Appunti personali e quelli di Damiano Gollinucci presi a lezione
    
    \item Communication Systems Engineering, Second Edition, Proakis - Salehi ISBN 0130617938

\end{itemize}

Chiaraluce è un professore con la P maiuscola: è coinvolgente, esaustivo e chiaro. \newline 

Come dice a lezione, nel corso ci saranno delle salite e poi delle discese, un po' come la vita. \newline 

Venite a lezione, se potete, e chiedete almeno un ricevimento al prof: in tempo massimo una settimana avrete modo di chiedergli qualsiasi dubbio avete in mente. \newline 

Venite a lezione anche per le esercitazioni di Matlab, perchè si potranno capire alcuni argomenti, specialmente quelli finali: dal metodo "carta e penna" al metodo iterativo e di programmazione. \newline 

Sfruttate il ricevimento il più possibile, perchè, giustamente, a lezione non vi verranno tutti i tre-miliardi di dubbi sul corso, 
specialmente quelli in cui il prof lascia le slide senza spiegazioni e solo con la formula. \newline 

La cosa più importante dell'orale è come ci si arriva alla formula. \newline 

È un po' come la vita: l'importante è il viaggio, non dove si arriva. \newline 

È altamente consigliato studiare e superare prima l’esame di teoria dei segnali, o come si chiama adesso "Segnali Determinati e Aleatori", 
perchè non capirete nulla sul corso, ve lo posso assicurare mettendo la mano sul fuoco. \newline 

Ogni concetto scritto in queste dispense, fa riferimento al corso precedente, o come lo chiama il professore a "Telecomunicazioni 1", perchè questa è "Telecomunicazioni 2". \newline 

Se vuoi dare un'occhiata ai miei vecchi appunti del corso di Teoria dei Segnali: \\
\url{https://github.com/ciccio25/appunti-teoria-dei-segnali} \newline 

I concetti importanti degli altri corsi precedenti che non sono quelli di Chiaralce sono importanti ma non fondamentali: ci serviranno i concetti base. \newline 

Lascerò dei link a video e/o spiegazioni esterne per ulteriori approfondimenti. \newline 

Per qualsiasi domanda, se trovi qualche errore grammaticale o hai bisogno di chiarimenti, scrivimi a \href{mailto:rossini.stefano.appunti@gmail.com}{rossini.stefano.appunti@gmail.com} o contattami su Whatsapp se hai il mio numero di cellulare personale. \newline

Se trovi qualche errore e/o vuoi contribuire agli appunti, 
puoi aprire un Issue sulla repository degli appunti di GitHub. \newline 

Buono studio e buona lettura \newline

\newpage 





